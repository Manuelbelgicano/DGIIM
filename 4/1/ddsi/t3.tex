\documentclass[12pt,a4paper]{article}

%%%%%%%%%%%%%%%%%%%%%%%%%%%%%%%%%%%%%%%%%
%				PAQUETES				%
%%%%%%%%%%%%%%%%%%%%%%%%%%%%%%%%%%%%%%%%%
\usepackage[a4paper]{geometry} % Márgenes
\usepackage{fancyhdr} % Encabezados/pies de páginas
\usepackage{amsmath} % Matemáticas
\usepackage{amsfonts} % Letras caligráficas para matemáticas
\usepackage{mathtools} % Matemáticas extra
\usepackage{xcolor} % colores
\usepackage[xcolor]{mdframed} % Marcos
\usepackage{amsthm} % Teoremas
\usepackage{enumitem} % Opciones de personalización de listas
\usepackage{amssymb}

%%%%%%%%%%%%%%%%%%%%%%%%%%%%%%%%%%%%%%%%%
%			   COLORINES				%
%%%%%%%%%%%%%%%%%%%%%%%%%%%%%%%%%%%%%%%%%
\definecolor{ma_3}{HTML}{F1F1F1} % Fondo ejemplos
\definecolor{ma_1}{HTML}{000000} % Ejemplos, demostraciones
\definecolor{m_4}{HTML}{000000} % Ejemplos, demostraciones

%%%%%%%%%%%%%%%%%%%%%%%%%%%%%%%%%%%%%%%%%
%				  MARCOS				%
%%%%%%%%%%%%%%%%%%%%%%%%%%%%%%%%%%%%%%%%%
\mdfsetup{ % Configuración general de los marcos
  skipabove=1em, % Espacio sobre los marcos
  skipbelow=1em, % Espacio bajo los marcos
  innertopmargin=0.3em, % Margen interior superior
  innerbottommargin=1em, % Margen interior inferior
  splittopskip=2\topsep, % Espacio entre marcos
  usetwoside=false, % Diferenciar páginas
}

\mdfdefinestyle{marco_ejemplos}{ % Nombre del estilo
	linewidth=2pt, % Grosor de la línea
	linecolor=ma_1, % Color de la línea
	backgroundcolor=ma_3, % Color de fondo
	topline=false, % Línea arriba
	leftline=true, % Línea a la izquierda
	bottomline=false, % Línea abajo
	rightline=false,% Línea a la derecha
	leftmargin=0em, % Margen a la izquierda
	innerleftmargin=1em, % Margen interior a la izquierda
	innerrightmargin=1em, % Margen interior a la derecha
	rightmargin=0em, % Margen a la derecha
}

\mdfdefinestyle{marco_anotaciones}{ % Nombre del estilo
	linewidth=1pt, % Grosor de la línea
	linecolor=ma_1, % Color de la línea
	backgroundcolor=ma_3, % Color de fondo
	topline=true, % Línea arriba
	leftline=true, % Línea a la izquierda
	bottomline=true, % Línea abajo
	rightline=true,% Línea a la derecha
	leftmargin=0em, % Margen a la izquierda
	innerleftmargin=1em, % Margen interior a la izquierda
	innerrightmargin=1em, % Margen interior a la derecha
	rightmargin=0em, % Margen a la derecha
}

\newtheoremstyle{ejemplo} % Nombre del estilo
{} % Espacio por encima
{} % Espacio por debajo
{} % Estilo del cuerpo
{} % Indentación
{\color{m_4}\itshape\bfseries} % Estilo de la cabecera
{} % Símbolo tras la cabecera
{ } % Espacio tras la cabecera
{} % Especificación de la cabecera
\theoremstyle{ejemplo}
\newtheorem*{eje}{} % Comando para los ejemplos
\newtheoremstyle{algoritmo} % Nombre del estilo
{} % Espacio por encima
{} % Espacio por debajo
{} % Estilo del cuerpo
{} % Indentación
{\color{m_4}\bfseries} % Estilo de la cabecera
{:} % Símbolo tras la cabecera
{\newline} % Espacio tras la cabecera
{} % Especificación de la cabecera
\theoremstyle{algoritmo}
\newtheorem*{alg}{Algoritmo} % Comando para los ejemplos
\surroundwithmdframed[style=marco_ejemplos]{eje}
\surroundwithmdframed[style=marco_anotaciones]{alg}

%%%%%%%%%%%%%%%%%%%%%%%%%%%%%%%%%%%%%%%%%
%				MÁRGENES				%
%%%%%%%%%%%%%%%%%%%%%%%%%%%%%%%%%%%%%%%%%
\geometry{
	left=2.5cm,
	right=2.5cm,
	bottom=2.5cm
}

%%%%%%%%%%%%%%%%%%%%%%%%%%%%%%%%%%%%%%%%%
%		ENCABEZADO/PIE DE PAGINA		%
%%%%%%%%%%%%%%%%%%%%%%%%%%%%%%%%%%%%%%%%%
\setlength{\headheight}{14pt}
\pagestyle{fancy}
\fancyhf{}
\fancyhead[R]{Universidad de Granada}
\fancyhead[L]{Normalización}
\fancyfoot[C]{\thepage}

\setlength\parindent{0pt} % Tamaño de la sangría

\begin{document}

%%%%%%%%%%%%%%%%%%%%%%%%%%%%%%%%%%%%%%%%%
%				 TÍTULO 				%
%%%%%%%%%%%%%%%%%%%%%%%%%%%%%%%%%%%%%%%%%
\title{\Huge{\textbf{Normalización}}}
\author{\Large{\textit{Desarrollo y Diseño de Sistemas de Información}}}
\date{Manuel Gachs Ballegeer}
\maketitle

\subsection*{Dependencias}
Dada una relación $R$, con subconjuntos de atributos $\alpha$ y $\beta$.
\\

\textbf{Dependencia funcional:}

Se dice que $\alpha$ determina funcionalmente a $\beta$ y se denota $\alpha\to
\beta$ si y sólo si $\forall r$ instancia de $R$ y $\forall s,t\in r$, si
$s(\alpha)=r(\alpha)\implies s(\beta)=r(\beta)$.
\\

\textbf{Dependencia funcional completa:}

Se dice que $\alpha$ determina funcionalmente a $\beta$ de forma completa si y
solo si $\alpha\to\beta$ y $\nexists\gamma\subset\alpha$ tal que $\gamma\to
\beta$.
\\

\textbf{Dependencia trivial:}

$\alpha\to\beta$ tal que $\beta\subset\alpha$.

\subsection*{Descomposición sin pérdidas}
Sea $(R,r)$ una relación que se descompone en $(R_1,r_1)$ y $(R_2,r_2)$. Se dice
que una descomposición es sin pérdidas si y sólo si podemos recuperar $R$ 
mediante reunión natural. Esto es:
\begin{itemize}
	\item [] $R_1\cup R_2=R$
	\item [] $r_1\text{\texttt{ join }}r_2=r$
\end{itemize}

\subsection*{Teorema de Heath}
Sea $R$ una relación con atributos $A$,$B$ y $C$, donde se verifica $B\to C$.
Entonces, la descomposición en relaciones:
\begin{itemize}
	\item [] $R_1(A,B)$
	\item [] $R_2(B,C)$
\end{itemize}
es una descomposición sin pérdidas.

\subsection*{Axiomas de Amstrong}

\textbf{Reflexividad:} 

$\forall\alpha,\beta$ con $\beta\subset\alpha$ se verifica $\alpha\to\beta$.
\\

\textbf{Ampliación:}

$\forall\alpha,\beta,\gamma$ con $\alpha\to\beta$ se verifica $\alpha\gamma
\to\beta\gamma$.
\\

\textbf{Transitividad:}

$\forall\alpha,\beta,\gamma$ con $\alpha\to\beta$ y $\beta\to\gamma$, se verifica
$\alpha\to\gamma$.
\\

\textbf{Reglas:}
\begin{itemize}
	\item \textit{Unión:} $\forall\alpha,\beta,\gamma$ con $\alpha\to\beta$ y 
	$\alpha\to\gamma$, se verifica $\alpha\to\beta\gamma$.
	\item \textit{Descomposición:} $\forall\alpha,\beta,\gamma$ con $\alpha\to
	\beta\gamma$, se verifica $\alpha\to\beta$ y $\alpha\to\gamma$.
	\item \textit{Pseudotransitividad:} $\forall\alpha,\beta,\gamma,\delta$ con
	$\alpha\to\beta$ y $\beta\gamma\to\delta$ se verifica $\alpha\gamma\to\delta$.
\end{itemize}

\subsection*{Dependencia transitiva problemática}
Sea $R$ un esquema de relación, $F$ un conjunto de dependencias funcionales
asociado y $C\subset R$ una clave candidata de $R$. Se dice que $C\to\beta$ con
$\beta$ atributo no primo es una dependencia transitiva problemática si $
\exists\alpha\subset R$ tal que $\alpha$ no es superclave y 
$$(C\to\alpha)\in F\quad\quad(\alpha\to\beta)\in F$$

\subsection*{Cierres}

\textbf{Cierre de un conjunto de dependencias funcionales $F$:}

Se nota por $F^+$ y representa el conjunto de todas las dependencias 
funcionales que pueden deducirse de las dependencias funcionales de $F$ 
aplicando los Axiomas y las Reglas de Armstrong en una secuencia finita de pasos.
\\

\textbf{Cierre de un conjunto de atributos $\alpha$ en base a un conjunto de 
dependencias funcionales $F$:}

Se nota por $\alpha^+$ y representa el conjunto de todos los atributos que son 
determinados por los atributos de $\alpha$ en conjunto mediante dependencias 
de $F^+$. $A\in\alpha_F^+\iff\alpha\to A\in F^+$.

\begin{alg}
Cierre de un conjunto de atributos $\alpha$ en base a un conjunto de 
dependencias funcionales $F$:
\begin{enumerate}[noitemsep]
	\item Inicializamos $\alpha_F^+=\alpha$.
	\item Mientras que $\alpha_F^+$ cambie:
	
	\quad\quad Si $\beta\to\gamma$ y $\beta\subset\alpha_F^+$, entonces $
	\alpha_F^+=\alpha_F^+\cup\{\gamma\}$.
\end{enumerate}
\end{alg}

\subsection*{Atributos raros}
Sea una dependencia de la forma $\alpha A\to\beta$, se dice que $A$ es
raro respecto a $\alpha$ si y sólo si $A\in\alpha^+$, es decir, $A$ depende
funcionalmente de los atributos que la acompañan.

\subsection*{Dependencias redundantes}
Sea una dependencia $\alpha\to\beta$ elemento del conjunto de dependencias $F$.
$\alpha\to\beta$ es una dependencia redundante si se puede obtener a partir de
las demás dependencias. Esto es:
$$\alpha\to\beta\text{ redundante}\iff(\alpha\to\beta)\in(F\setminus\{\alpha
\to\beta\})^+\iff\beta\in\alpha^+_{F\setminus\{\alpha\to\beta\}}$$

\subsection*{Segunda forma normal (2FN)}
Decimos que una relación $R$ está en 2FN si y sólo si:
\begin{itemize}[noitemsep]
	\item Está en primera forma normal (1FN).
	\item Todos sus atributos no primos dependen funcionalmente de forma 
	completa de las claves candidatas.
\end{itemize}
\textit{(Suele darse en tablas generadas por un buen diseño conceptual)}

\subsection*{Tercera forma normal (3FN)}
Una relación $R$ está en tercera forma normal si y sólo si:
\begin{itemize}[noitemsep]
	\item Está en segunda forma normal (2FN).
	\item No presenta dependencias transitivas problemáticas.
\end{itemize}

\subsection*{Forma normal de Boyce-Codd (FNBC)}
Decimos que una relación $R$ está en FNBC si y sólo si para toda dependencia no
trivial $\alpha\to\beta$ se cumple que $\alpha$ es una superclave. Una relación
en FNBC está en 3FN, y una relación en 3FN con una única clave candidata está en
FNBC.

\begin{alg}
Dada una relación $R$ y su conjunto de dependencias $F$, el conjunto $T$ 
resultante de relaciones en FNBC se calcula:
\begin{enumerate}[noitemsep]
	\item $T=\{R\}$
	\item Si existe $(\alpha\to\beta)\in F$ no trivial con $\alpha,\beta\subset 
	R$ tal que $\alpha$ no es superclave:

	\quad\quad $T=T-\{R\}$
	
	\quad\quad Se divide $R$ en $R_1$ con esquema $R\setminus\beta$ y $R_2$ con
	esquema $\alpha\cup\beta$.
	
	\quad\quad $T=T\cup\text{FNBC}(R_1)$
	
	\quad\quad $T=T\cup\text{FNBC}(R_2)$
	
\end{enumerate}
Es importante saber que se deben conocer las claves candidatas de $R$ (y $R_1$, 
$R_2$, ...).
\end{alg}

El proceso de normalización de una relación cualquiera $R$ a su equivalente en
FNBC tiene el inconveniente de que puede eliminar dependencias funcionales, 
puesto que se basa en separar la relación $R$ en un conjunto de relaciones $R_i$.
\\

A la hora de elegir la dependencia $\alpha\to\beta$ con la que realizar la 
separación que minimice el número de dependencias funcionales perdidas, se sigue 
el siguiente criterio con respecto a $\beta$:
\begin{enumerate}[noitemsep]
	\item $\beta$ es un atributo que sólo aparece a la derecha de las 
	dependencias.
	\item $\beta$ es un atributo que no participa en una clave candidata.
	\item $\beta$ es el atributo que participa en el menor número de dependencias.
\end{enumerate}

\subsection*{Cálculo de claves candidatas}
Dado un esquema $R$ con su conjunto de relaciones $F$, el conjunto de claves
candidatas se calcula de la siguiente manera:
\\

\textbf{1. Eliminación de atributos independientes.}

A partir de $R$, se construye $R_{si}$, que contiene todos los atributos de $R$
que no son independientes, ya que participan en cualquier clave candidata y no
se pueden deducir otros a partir de ellos.
\\

\textbf{2. Eliminación de atributos equivalentes.}

A partir de $R_{si}$, se construye $R_{sie}$, en el que se eliminan los 
atributos equivalentes. Por cada equivalencia, se elimina uno de los dos y se
sustituye el eliminado por su equivalencia en las dependencias correspondientes.
Algunos atributos pueden transformarse en independientes.
\\

\textbf{3. Selección de una clave de $R_{sie}$ de determinantes no 
determinados.}

Se selecciona como primer candidato a la clave candidata $K_p$ cualquier
determinante de $R_{sie}$ que no sea determinado.

Si es el único determinante, $K_p$ es clave candidata y se pasa al punto 
\textbf{5}.
\\

\textbf{4. Selección de claves de $R_{sie}$ con posibles determinantes  
determinados.}

Se construye $R_{sie}'$, eliminando de $R_{sie}$ los atributos de $K_p^+$ no
implicados en el cálculo de $K_p^+$. Si no puede construirse $R_{sie}'$, 
entonces se procede con $R_{sie}'=R_{sie}$.

Para cada determinante $\alpha$ que es determinado, se construye $K_p'$ clave provisional de $R_{sie}'$, con $K_p'=K_p\cup\{\alpha\}$. Si ${K'}_{p}^{+}=
R_{sie}'$, entonces $K_p'$ es clave de $R_{sie}'$. En caso contrario, se añade
otro determinante determinado que no pertenezca a ${K'}_{p}^{+}$ y se vuelve 
a comprobar la igualdad.

Se añaden a $K_p$ todas las claves de $R_{sie}'$ para obtener las claves de 
$R_{sie}$.
\\

\textbf{5. Adición de los atributos independientes a las claves de $R_{sie}$.}
\\

\textbf{6. Réplicar las claves con las equivalencias eliminadas en el paso 2.}

Esto es, si existe la equivalencia $\alpha\equiv\beta$ y $\alpha A$ es una clave
candidata, se añade como clave candidata $\beta A$.

\subsection*{Recubrimiento minimal o canónico}
Se dice que $F'$ es el recubrimiento minimal o canónico del conjunto de
dependencias $F$ al más simple de los conjuntos que cumplen $F^+={F'}^+$.

\begin{alg}
Los pasos para calcular el recubrimiento canónico de un conjunto de dependencias
$F$ es el siguiente:
\begin{enumerate}[noitemsep]
	\item Construimos $F^{(1)}$ descomponiendo todas las dependencias con parte
	derecha compuesta.
	\item Construimos $F^{(2)}$ a partir de $F^{(1)}$ eliminando todos los
	atributos raros.
	\item Obtenemos el cierre canónico $F'$ a partir de $F^{(2)}$ eliminando
	todas las dependencias redundantes.
\end{enumerate}
\end{alg}

\end{document}