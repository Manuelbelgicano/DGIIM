\documentclass[11pt,titlepage,a4paper]{article}
%%%%%%%%%%%%%%%%%%%%%%%%%%%%%%%%%%%%%%%%%
%				PAQUETES				%
%%%%%%%%%%%%%%%%%%%%%%%%%%%%%%%%%%%%%%%%%
\usepackage{fancyhdr}
\usepackage{amsmath}
\usepackage{mathtools}
\usepackage{amsfonts}

%%%%%%%%%%%%%%%%%%%%%%%%%%%%%%%%%%%%%%%%%
%				MÁRGENES				%
%%%%%%%%%%%%%%%%%%%%%%%%%%%%%%%%%%%%%%%%%
\addtolength{\oddsidemargin}{-1in}
\addtolength{\evensidemargin}{-1in}
\addtolength{\textwidth}{+1.75in}
\addtolength{\topmargin}{-0.675in}
\addtolength{\textheight}{1.25in}

%%%%%%%%%%%%%%%%%%%%%%%%%%%%%%%%%%%%%%%%%
%				COMANDOS				%
%%%%%%%%%%%%%%%%%%%%%%%%%%%%%%%%%%%%%%%%%
\renewcommand{\contentsname}{Índice}

%%%%%%%%%%%%%%%%%%%%%%%%%%%%%%%%%%%%%%%%%
%		ENCABEZADO/PIE DE PAGINA		%
%%%%%%%%%%%%%%%%%%%%%%%%%%%%%%%%%%%%%%%%%
\setlength{\headheight}{14pt}
\pagestyle{fancy}
\fancyhf{}
\fancyhead[RE,RO]{Análisis Matemático I}
\fancyhead[LE,LO]{Curso 2018-2019}
\fancyfoot[CE,CO]{\thepage}

%%%%%%%%%%%%%%%%%%%%%%%%%%%%%%%%%%%%%%%%%
%				DOCUMENTO				%
%%%%%%%%%%%%%%%%%%%%%%%%%%%%%%%%%%%%%%%%%
\begin{document}

%%%%%%%%%%%%%%%%%%%%%%%%%%%%%%%%%%%%%%%%%
%				 TÍTULO 				%
%%%%%%%%%%%%%%%%%%%%%%%%%%%%%%%%%%%%%%%%%
\title{\Huge{\textbf{Análisis Matemático I: Una breve exposición de los conceptos}}}
\author{\textit{Manuel Gachs Ballegeer}}
\date{enero 2019}
\maketitle

%%%%%%%%%%%%%%%%%%%%%%%%%%%%%%%%%%%%%%%%%
%				 ÍNDICE 				%
%%%%%%%%%%%%%%%%%%%%%%%%%%%%%%%%%%%%%%%%%
\tableofcontents
\clearpage

\section{Espacios métricos}
\subsection{Subconjuntos}
Sea $(E,d)$ un espacio métrico y $A\subset(E,d)$. Se definen los siguientes subconjuntos de $A$:
\begin{itemize}
\item Adherencia de $A$ (\=A): 
$\forall a\in\text{\=A},\exists r>0:B(a,r)\cap A\neq\emptyset$
\item Interior de A (\AA):
$\forall a\in\text{\AA},\exists r>0:B(a,r)\subset A$
\item Frontera de A (Fr($A$)):
$\text{Fr}(A):=\text{\AA}\setminus\text{\=A}$
\item Puntos de acumulación ($A'$):
$\forall a\in A',\exists r>0:B(a,r)\cap(A\setminus\{a\})\neq\emptyset$ 
\item Puntos aislados: 
Se dice que un punto $a\in A$ es aislado si $\exists r>0$ tal que $B(a,r)\cap A=\{a\}$
\end{itemize}

\subsubsection*{Implicaciones entre los subconjuntos}
Sea $(E,d)$ un espacio métrico y $A\subset(E,d)$. Existen las siguientes relaciones:
\begin{itemize}
\item $A\subset\text{\=A}$ . Además, \=A es el cerrado más pequeño que contiene a $A$.
\item $a\in\text{\=A}\iff$ existe una sucesión en A que converge a $a$
\item $A\supset\text{\AA}$ . Además, \AA\ es el abierto más grande contenido en $A$.
\item $\forall a\in\text{\=A}$ , $a\in A'$ ó $a$ es un punto aislado $\iff\forall a\in A$ , $a\in A'$ ó $a$ es un punto aislado
\end{itemize}

\subsection{Complitud, compacidad, convexidad}
\subsubsection*{Complitud}
Sea $(E,d)$ un espacio métrico. Una sucesión $\{x_n\}\in(E,d)$ es de Cauchy si:
\begin{equation*}
\forall\varepsilon>0,\exists m\in\mathbb{N}: p,q\geq m\implies d(x_p,x_q)<\varepsilon
\end{equation*}
Se dice que un espacio métrico es \emph{completo} si toda sucesión de Cauchy es convergente. Si además es normado, se le llama \emph{espacio de Banach}.
\begin{itemize}
\item Espacio completo de un espacio métrico $\implies$ espacio cerrado
\item Espacio cerrado de un espacio métrico completo $\implies$ espacio completo
\end{itemize}

\subsubsection*{Compacidad}
Toda sucesión de $\mathbb{R}^N$ admite una parcial convergente. Además, las normas son equivalentes.
Se dice que un subconjunto $K\subset\mathbb{R}^N$ es \emph{compacto} si es cerrrado y acotado.
\begin{equation*}
K\text{ es compacto}\iff \text{toda sucesión de puntos de }K\text{ converge a un punto de }K
\end{equation*}

\subsubsection*{Convexidad y conexidad}
Se dice que un subconjunto $K\subset\mathbb{R}^N$ es \emph{convexo} si se verifica
\begin{equation*}
a,b\in K\implies a+t(b-a)\in K,\forall t\in[0,1]
\end{equation*}
es decir, si el segmento de extremos $a$ y $b$ está en $K$.\\\\
Se dice que un subconjunto $C$ de un espacio métrico es \emph{conexo} si 
\begin{equation*}
O_1,O_2 \text{ abiertos de } C: O_1\cap O_2=\emptyset\text{ y }
O_1\cup O_2=\emptyset\implies O_1=\emptyset\text{ ó }O_2=\emptyset
\end{equation*}

\section{Funciones}
Sean $M,N\in\mathbb{N}$ y $A\in\mathbb{R}^N$.
\begin{itemize}
\item Un \emph{campo escalar} en $A$ es una función $f:A\to\mathbb{R}$
\item Un \emph{campo vectorial} en $A$ es una función 
$f=(f_1,\dotsc,f_M):A\to\mathbb{R}^N$, donde cada $f_i$ es un campo escalar de $f$ o función componente de $f$.
\end{itemize}

\subsection{Continuidad}
Se dice que un campo escalar es continuo en un punto $a$ si
\begin{equation*}
[\forall a\in A,\{a_n\}\to a]\implies\{f(a_n)\}\to f(a)
\end{equation*}
En caso de campos vectoriales, se dice que $f$ es continuo en $a\iff f_i$ es continuo en $a$, $\forall i=1,\dotsc, M$\\\\
Sean $g,f$ funciones continuas en $a$. Entonces $g\circ f$ es continua en $a$.\\\\
Sean $(E,d_1)$ y $(F,d_2)$ espacios métricos, $A\subset E$ y $f:A\to F$ una función. $f$ es \emph{uniformemente continua} si
\begin{equation*}
\forall\varepsilon>0,\exists\delta>0:[x,y\in A,d_1(x,y)<\delta]\implies d_2(f(x),f(y))<\varepsilon
\end{equation*}

\subsection{Conservación de la compacidad y la conexidad}
\subsubsection*{Compacidad}
Sean $E,F$ espacios métricos, $K\subset E$ un subconjunto compacto y $f:K\to F$ continua. Entonces
\renewcommand{\theenumi}{\roman{enumi}}%
\begin{enumerate}
\item $f(K)$ es compacto.
\item Si $F\subset\mathbb{R}$, entonces $f$ es acotada y alcanza su máximo y mínimos absolutos.
\item $f$ es uniformemente continua.
\end{enumerate}

\subsubsection*{Conexidad}
Sean $E,F$ espacios métricos, $K\subset E$ un subconjunto conexo no vacío y $f:C\to F$ continua. 
Entonces $f(C)$ es conexo.\\
Los segmentos de $\mathbb{R}^N$ son conexos. Además, todo convexo de $\mathbb{R}^N$ es conexo.\\
Toda función continua de $C$ en $\{0,1\}$ es constante.

\section{Límites}












\end{document}