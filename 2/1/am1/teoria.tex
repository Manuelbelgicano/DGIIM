\documentclass[11pt,titlepage,a4paper]{article}
%%%%%%%%%%%%%%%%%%%%%%%%%%%%%%%%%%%%%%%%%
%				PAQUETES				%
%%%%%%%%%%%%%%%%%%%%%%%%%%%%%%%%%%%%%%%%%
\usepackage{fancyhdr}
\usepackage[spanish]{babel}
\usepackage{amsmath}
\usepackage{mathtools}
\usepackage{amsfonts}
\usepackage{amsthm}

%%%%%%%%%%%%%%%%%%%%%%%%%%%%%%%%%%%%%%%%%
%				MÁRGENES				%
%%%%%%%%%%%%%%%%%%%%%%%%%%%%%%%%%%%%%%%%%
\addtolength{\oddsidemargin}{-1in}
\addtolength{\evensidemargin}{-1in}
\addtolength{\textwidth}{+1.75in}
\addtolength{\topmargin}{-0.675in}
\addtolength{\textheight}{1.25in}

%%%%%%%%%%%%%%%%%%%%%%%%%%%%%%%%%%%%%%%%%
%				COMANDOS				%
%%%%%%%%%%%%%%%%%%%%%%%%%%%%%%%%%%%%%%%%%
\renewcommand{\contentsname}{Índice}
\theoremstyle{definition}
\newtheorem*{teo}{Teorema}
\newtheorem*{prop}{Proposición}

%%%%%%%%%%%%%%%%%%%%%%%%%%%%%%%%%%%%%%%%%
%		ENCABEZADO/PIE DE PAGINA		%
%%%%%%%%%%%%%%%%%%%%%%%%%%%%%%%%%%%%%%%%%
\setlength{\headheight}{14pt}
\pagestyle{fancy}
\fancyhf{}
\fancyhead[RE,RO]{Análisis Matemático I}
\fancyhead[LE,LO]{Curso 2018-2019}
\fancyfoot[CE,CO]{\thepage}

%%%%%%%%%%%%%%%%%%%%%%%%%%%%%%%%%%%%%%%%%
%				DOCUMENTO				%
%%%%%%%%%%%%%%%%%%%%%%%%%%%%%%%%%%%%%%%%%
\begin{document}

%%%%%%%%%%%%%%%%%%%%%%%%%%%%%%%%%%%%%%%%%
%				 TÍTULO 				%
%%%%%%%%%%%%%%%%%%%%%%%%%%%%%%%%%%%%%%%%%
\title{\Huge{\textbf{Análisis Matemático I: Temas de teoría}}}
\author{\textit{Marta Gómez}\\ \textit{Manuel Gachs Ballegeer\thanks{Transcripción a \LaTeX}}}
\date{diciembre 2018}
\maketitle

%%%%%%%%%%%%%%%%%%%%%%%%%%%%%%%%%%%%%%%%%
%				 ÍNDICE 				%
%%%%%%%%%%%%%%%%%%%%%%%%%%%%%%%%%%%%%%%%%
\tableofcontents
\clearpage

%%%%%%%%%%%%%%%%%%%%%%%%%%%%%%%%%%%%%%%%%
%				 TEMA 1º 				%
%%%%%%%%%%%%%%%%%%%%%%%%%%%%%%%%%%%%%%%%%
\section{Complitud}

Dado un espacio métrico $(E,d)$, se dice que $d$ es una distancia completa, 
o que $E$ es un espacio métrico completo cuando toda sucesión de Cauchy de 
puntos de $E$ es convergente.

\subsubsection*{Sucesiones de Cauchy}

Sea $(E,d)$ un espacio métrico y $\{x_n\}\in E,\forall n\in\mathbb{N}$, se 
dice que $\{x_n\}$ es una sucesión de Cauchy en $E$ cuando:
\begin{equation*}
\forall\varepsilon>0\quad\exists m\in\mathbb{N}:p,q\geq m\implies d(x_p,x_q)<\varepsilon
\end{equation*}
En cualquier espacio métrico, toda sucesión convergente es una sucesión de Cauchy.
\\\\
El teorema de complitud en $\mathbb{R}$ afirma que la distancia usual de 
$\mathbb{R}$ es completa, o que $\mathbb{R}$ con la distancia usual es un
espacio métrico completo. La complitud de un espacio métrico no es una 
propiedad topológica.
\\\\
Una norma $\|\cdot\|$ en un espacio vectorial $X$ es una norma completa cuando
es completa la distancia $d$ asociada, definida por $d(x,y)=\|y-x\|\forall x,y\in X$.
Un espacio normado cuya norma es completa recibe el nombre de espacio de Banach.

\subsection{Teorema del punto fijo de Banach}

\begin{teo}
Sea $E$ un espacio métrico completo no vacío y $f:E\to E$ una aplicación contractiva.
Entonces $f$ tiene un único punto fijp, esto es, existe un único punto $x\in E$
tal que $f(x)=x$.
\end{teo}
\begin{proof}
Fijamos $x_0\in E$ arbitrario y definimos por inducción una sucesión $\{x_n\}$ de
puntos de $E$, tomando $x_1=f(x_0)$ y $x_{n+1}=f(x_n)$, $\forall n\in\mathbb{N}$.
Probaremos que $\{x_n\}$ es convergente y su límite será el punto fijo que buscamos.
\\
Si $\alpha>1$ es la constante de Lipschitz de $f$ y $p=d(x_0,x_1)$, comprobamos
por inducción la siguiente desigualdad:
\begin{equation}\label{eq:suspuestoinduccion}
d(f(x_0),f(x_1))\leq\alpha^np,\forall n\in\mathbb{N}
\end{equation}
\\
En efecto, tenemos $d(x_1,x_2)=d(f(x_0),f(x_1))\leq\alpha d(x_n,x_{n+1})$ y, 
suponiendo que \eqref{eq:suspuestoinduccion} se verifica para un $n\in\mathbb{N}$,
deducimos que $d(x_{n+1},x_{n+2})=d(f(x_n),f(x_{n+1}))\leq\alpha d(x_n,x_{n+1})
\leq\alpha\alpha^np=\alpha^{n+1}p$.
\\
Ahora, para cualesquiera $n,k\in\mathbb{N}$, tenemos:
\begin{equation}\label{eq:desigualdad}
d(x_n,x_{n+k})\leq\sum_{j=0}^{k-1}d(x_{n+j},x_{n+j+1})\leq p\sum_{j=0}^{k-1}\alpha^{n+j}
\leq p\alpha^n\sum_{j=0}^{\infty}\alpha^j=\frac{p\alpha^n}{1-\alpha}
\end{equation}
\\
De la desigualdad \eqref{eq:desigualdad} deducimos que $\{x_n\}$ es una sucesión de
Cauchy. Dado $\varepsilon>0$, como $\{\alpha^n\}\to 0$, $\exists m\in\mathbb{N}$ tal que,
paran $n\geq m$ se tiene $p\alpha^n<\varepsilon(1-\alpha)$. Entonces, para $p,p\geq m$,
suponiendo que $p<q$, usamos \eqref{eq:desigualdad} con $n=p$ y $k=q-p:d(x_n,x_{n+k})
\leq\frac{p\alpha^n}{1-\alpha}<\varepsilon$.
\\
Como por hipótesis $E$ es completo, tenemos $\{x_n\}\to x\in E$, luego $\{f(x_n)\}
=\{x_{n+1}\}\to x$. Pero $f$ es continua, luego $\{f(x_n)\}\to f(x)$, con lo que concluimos
que $f(x)=x$. Para comprobar que es único, tomamos $y$ como otro punto fijo. Se tiene
$d(x,y)=d(f(x),f(y))\leq\alpha d(x,y)$, luego $(1-\alpha)d(x,y)\leq 0$. Como $\alpha>1$,
deducimos que $d(x,y)\leq 0$, luego $x=y$.
\end{proof}

%%%%%%%%%%%%%%%%%%%%%%%%%%%%%%%%%%%%%%%%%
%				 TEMA 2º 				%
%%%%%%%%%%%%%%%%%%%%%%%%%%%%%%%%%%%%%%%%%
\section{Compacidad}

Se dice que un espacio métrico $E$ es compacto cuando toda sucesión de puntos de 
$E$ admite una sucesión parcial convergente. Se trata de una propiedad topológica.
\\

Si $E$ es un espacio métrico y $A\subset E$, diremos que $A$ es un subconjunto
compacto de $E$ cuando $A$ sea un espacio métrico compacto con la distancia inducida
por la de $E$. Esto significa que toda sucesión de puntos de $A$ admite una parcial
convergente a un punto de $A$.

\subsection{Caracterización de la compacidad en \textbf{$\mathbb{R}^N$}}

\begin{teo}
(Teorema de Bolzano-Weierstrass) Toda sucesión acotada de vectores de $\mathbb{R}^N$
admite una sucesión parcial convergente.
\end{teo}
\begin{prop}
Sea $A$ un subconjunto compacto de un espacio métrico $E$. Entonces $A$ esta acotado y es
un subconjunto cerrado de $E$.
\end{prop}
\begin{proof}
Suponiendo que $A$ no está contenido en ninguna bola llegarenos a una contradicción.
Fijado un punto cualquiera $x\in E$, $\forall n\in\mathbb{N}$, $\exists x_n\in A$ tal que
$d(x_n,x)<n$. Entonces $\{x_n\}$ admite una sucesión parcial $\{x_{\sigma(n)}\}$ que
converge a un punto $a\in A$, es decir, $\{d(x_{\sigma(n)},a)\}\to 0$. Pero también
$d(x_{\sigma(n)},x)\leq d(x_{\sigma(n)},a)+d(x,a),\forall n\in\mathbb{N}$, luego la 
sucesión $\{d(x_{\sigma(n)},x)\}$ está mayorada, lo cual es una contradicción, ya que
$d(x_{\sigma(n)},x)\geq\sigma(n)\geq n,\forall n\in\mathbb{N}$.
\\\\
Sea ahora $x\in\text{\=A}$ y $\{a_n\}$ una sucesión de puntos de $A$ con $\{a_n\}\to x$.
Por ser $A$ compacto, tenemos una sucesión parcial $\{a_{\sigma(n)}\}$ que converge a
un $a\in A$, pero también $\{a_{\sigma(n)}\}\to x$, luego $x=a\in A$. Esto prueba que
$\text{\=A}\subset A$, es decir, $A$ es cerrado.
\end{proof}
\begin{prop}
Un subconjunto de $\mathbb{R}^N$ es compacto si, y sólo si es cerrado y acotado.
\end{prop}
\begin{proof}
Ya hemos visto que una implicación es válida, no sólo en $\mathbb{R}^N$, sino en cualquier
espacio métrico. Para el recíproco, sea $A$ un conjunto cerrado y acotado de $\mathbb{R}^N$.
El toerema de Bolzano-Weierstrass nos dice que $\{a_n\}$ admite una sucesión parcial
$\{a_{\sigma(n)}\}$ que converge a un vector $x\in\mathbb{R}^N$. Como $A$ es cerrado, tenemos
$x\in A$, lo que prueba que toda sucesión de puntos de $A$ admite una sucesión parcial
que converge a un punto de $A$, es decir, $A$ es compacto.
\end{proof}

\subsection{Teorema de la conservación de la compacidad}

\begin{teo}
Sean $E$ y $F$ dos espacios métricos y $f:E\to F$ una función continua. Si $E$ es compacto, 
entonces $f(E)$ es compacto.
\end{teo}
\begin{proof}
Dada una sucesión $\{y_n\}$ de puntos de $f(E)$, deberemos probar que $\{y_n\}$ admite
una sucesión parcial que converge a un punto de $f(E)$.
\end{proof}

\end{document}