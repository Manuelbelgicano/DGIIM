\documentclass[11pt,titlepage,a4paper]{article}
%%%%%%%%%%%%%%%%%%%%%%%%%%%%%%%%%%%%%%%%%
%				PAQUETES				%
%%%%%%%%%%%%%%%%%%%%%%%%%%%%%%%%%%%%%%%%%
\usepackage{fancyhdr}
\usepackage{amsmath}
\usepackage{mathtools}
\usepackage{amsfonts}

%%%%%%%%%%%%%%%%%%%%%%%%%%%%%%%%%%%%%%%%%
%				MÁRGENES				%
%%%%%%%%%%%%%%%%%%%%%%%%%%%%%%%%%%%%%%%%%
\addtolength{\oddsidemargin}{-1in}
\addtolength{\evensidemargin}{-1in}
\addtolength{\textwidth}{+1.75in}
\addtolength{\topmargin}{-0.675in}
\addtolength{\textheight}{1.25in}

%%%%%%%%%%%%%%%%%%%%%%%%%%%%%%%%%%%%%%%%%
%				COMANDOS				%
%%%%%%%%%%%%%%%%%%%%%%%%%%%%%%%%%%%%%%%%%
\renewcommand{\contentsname}{Índice}

%%%%%%%%%%%%%%%%%%%%%%%%%%%%%%%%%%%%%%%%%
%		ENCABEZADO/PIE DE PAGINA		%
%%%%%%%%%%%%%%%%%%%%%%%%%%%%%%%%%%%%%%%%%
\setlength{\headheight}{14pt}
\pagestyle{fancy}
\fancyhf{}
\fancyhead[RE,RO]{Análisis Matemático I}
\fancyhead[LE,LO]{Curso 2018-2019}
\fancyfoot[CE,CO]{\thepage}

%%%%%%%%%%%%%%%%%%%%%%%%%%%%%%%%%%%%%%%%%
%				DOCUMENTO				%
%%%%%%%%%%%%%%%%%%%%%%%%%%%%%%%%%%%%%%%%%
\begin{document}

%%%%%%%%%%%%%%%%%%%%%%%%%%%%%%%%%%%%%%%%%
%				 TÍTULO 				%
%%%%%%%%%%%%%%%%%%%%%%%%%%%%%%%%%%%%%%%%%
\title{\Huge{\textbf{Análisis Matemático I: Temas de teoría}}}
\author{\textit{Manuel Gachs Ballegeer}}
\date{diciembre 2018}
\maketitle

%%%%%%%%%%%%%%%%%%%%%%%%%%%%%%%%%%%%%%%%%
%				 ÍNDICE 				%
%%%%%%%%%%%%%%%%%%%%%%%%%%%%%%%%%%%%%%%%%
\tableofcontents
\clearpage

%%%%%%%%%%%%%%%%%%%%%%%%%%%%%%%%%%%%%%%%%
%				COMPLITUD				%
%%%%%%%%%%%%%%%%%%%%%%%%%%%%%%%%%%%%%%%%%
\section{Complitud}
\subsubsection*{Sucesiones de Cauchy}
En espacios métricos, las sucesiones de Cauchy se definen de la siguiente manera: Si $E$ es un espacio 
métrico con una distancia $d$, y $x_n\in E$ para todo $n\in \mathbb{N}$, se dice que \{$x_n$\} es una sucesión de Cauchy 
en $E$, cuando:
\begin{equation*}
\forall\epsilon>0\quad\exists m\in\mathbb{N}:p,q\geq m\implies d(x_p,x_q)<\epsilon
\end{equation*}
Tenemos además la siguiente implicación:
\begin{center}
\textit{En cualquier espacio métrico, toda sucesión convergente es una sucesión de Cauchy.}
\end{center}
Para probarlo, sea \{$x_n$\} una sucesión de puntos del espacio métrico $E$, y supongamos que 
\{$x_n$\}$\rightarrow x\in E$. Dado $\epsilon>0$, existe $m\in\mathbb{N}$ tal que, para $n\geq m$ se tiene 
$d(x_n,x)<^\epsilon/_2$. Para $p,q\geq m$, concluimos entonces que $d(x_p,x_q)\leq d(x_p,x)+d(x,x_q)<\epsilon$.

\subsection{Teorema del punto fijo de Banach}
\section{Compacidad}
\subsection{Caracterización de la compacidad en \textbf{$\mathbb{R}^N$}}
\subsection{Teorema de la conservación de la compacidad}
\section{Espacios normados}
\subsection{Normas equivalentes}
\subsection{Teorema de Hausdorff}
\section{Conexión}
\subsection{Caracterización y conservación mediante funciones continuas}
\section{Diferenciabilidad}
\subsection{Derivadas direccionales}
\subsection{Derivadas parciales}
\subsection{Relaciones entre derivadas direccionales y parciales}
\section{Teorema del valor medio}
\subsection{Consecuencias}
\section{Teorema de Taylor para campos escalares}
\subsection{Aplicaciones}
\section{Teorema de la función inversa y de la función implícita}
\end{document}