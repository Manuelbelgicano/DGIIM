\documentclass[11pt,twoside,a4paper]{article}
%%%%%%%%%%%%%%%%%%%%%%%%%%%%%%%%%%%%%%%%%
%				PAQUETES				%
%%%%%%%%%%%%%%%%%%%%%%%%%%%%%%%%%%%%%%%%%
\usepackage{amsmath}
\usepackage{amsfonts}
\usepackage{enumitem}
\usepackage{fancyhdr}
\usepackage[a4paper]{geometry}
\usepackage{mathtools}
\usepackage{scrextend}
\usepackage{sansmathfonts}
\usepackage[T1]{fontenc}

%%%%%%%%%%%%%%%%%%%%%%%%%%%%%%%%%%%%%%%%%
%				MÁRGENES				%
%%%%%%%%%%%%%%%%%%%%%%%%%%%%%%%%%%%%%%%%%
\geometry{
	left=2.5cm,
	right=2.5cm,
	bottom=2.5cm
}

%%%%%%%%%%%%%%%%%%%%%%%%%%%%%%%%%%%%%%%%%
%				COMANDOS				%
%%%%%%%%%%%%%%%%%%%%%%%%%%%%%%%%%%%%%%%%%
\renewcommand{\contentsname}{\Huge{Índice}} %Cambiar el título del índice
\renewcommand*\familydefault{\sfdefault} %Cambiar la tipografía
\renewcommand{\arraystretch}{1.3} %Cambiar el tamaño entre líneas de una pabla

%%%%%%%%%%%%%%%%%%%%%%%%%%%%%%%%%%%%%%%%%
%		ENCABEZADO/PIE DE PAGINA		%
%%%%%%%%%%%%%%%%%%%%%%%%%%%%%%%%%%%%%%%%%
\setlength{\headheight}{14pt}
\pagestyle{fancy}
\fancyhf{}
\fancyhead[RE,RO]{Curso 2018-2019}
\fancyhead[LE,LO]{AC. Tema 1}
\fancyfoot[CE,CO]{\thepage}

\begin{document}

%%%%%%%%%%%%%%%%%%%%%%%%%%%%%%%%%%%%%%%%%
%				 TÍTULO 				%
%%%%%%%%%%%%%%%%%%%%%%%%%%%%%%%%%%%%%%%%%
\title{\Huge{\textbf{AC. Tema 1. Tipos de paralelismo}}}
\author{\textit{Doble Grado en Ingeniería Informática y Matemáticas}}
\date{2019}
\maketitle

%%%%%%%%%%%%%%%%%%%%%%%%%%%%%%%%%%%%%%%%%
%				 ÍNDICE 				%
%%%%%%%%%%%%%%%%%%%%%%%%%%%%%%%%%%%%%%%%%
%\tableofcontents
%\clearpage

%%%%%%%%%%%%%%%%%%%%%%%%%%%%%%%%%%%%%%%%%
%				 LECCION 1 				%
%%%%%%%%%%%%%%%%%%%%%%%%%%%%%%%%%%%%%%%%%
\section{Niveles y tipos de paralelismo implícito en aplicaciones}
La mayoría de computadores actuales implementan paralelismo en varios niveles de arquitectura. En una
aplicación se pueden distinguir distintos niveles de paralelismo, que aprovechan distintas partes del
computador. Cada nivel corresponde a un nivel de abstracción distinto de un código secuencial. Los distintos
niveles son los siguientes:
\begin{itemize}[noitemsep]
	\item \textit{Nivel de programas}. Los diferentes programas que intervienen en una aplicación o en 
	diferentes aplicaciones que se ejecutan en paralelo. La dependencia entre varios programas es poco  
	probable.
	\item \textit{Nivel de funciones}. Las funciones que componen un programa. Se pueden ejecutar en
	paralelo siempre que no exista dependencias inevitables entre ellas.
	\item \textit{Nivel de bucles (bloques)}. Una función puede tener en su interior varios bucles. El
	código dentro de un bucle se ejecuta varias veces, por lo que se pueden ejecutar las iteraciones en 
	paralelo, siempre que no existan dependencias entre las iteraciones.
	\item \textit{Nivel de operaciones}. En este nivel se encuentran las operaciones y las instrucciones de
	un código. Las operaciones independientes se pueden ejecutar en paralelo. Además, en algunos 
	procesadores, algunas instrucciones se componen de varias operaciones que pueden ejecutarse en paralelo.
\end{itemize}

Existe otra forma de clasificar el paralelismo, que es en función de la \textit{granularidad}
\footnote{Maginitud de la tarea (número de operaciones).}, que se suele corresponder con los niveles de
paralelismo. El \textit{grano fino} corresponde con el nivel de operaciones, mientras que el nivel de
programas se corresponde con el \textit{grano grueso}. Entre ambos encontramos el \textit{grano medio}, al
que se le asocia el nivel de funciones.

\subsubsection*{Tipos de paralelismo}

Existen tres tipos esenciales de paralelismo: El paralelismo funcional, el paralelismo de datos y el
paralelismo de tareas.
\begin{itemize}[noitemsep]
	\item \textit{Paralelismo funcional}. Se trata del paralelismo por niveles descrito anteriormente.
	\item \textit{Paralelismo de tareas}. Este paralelismo se encuentra extrayendo la estructura lógica de
	las funciones de una aplicación. En esta estructura, las funciones son bloques y los flujos de datos
	entre ellas son reflejados por conexiones. Es equivalente al nivel de función.
	\item \textit{Paralelismo de datos}. Este tipo de paralelismo se encuentra implícito en las operaciones
	con estructuras de datos (matrices, vectores, \dots). Se puede extraer mediante la representación
	matemática de las operaciones de la aplicación. Al implementarse mediante bucles las operaciones con
	estructuras de datos, éstas pueden realizarse en paralelo, siempre y cuando no existan dependencias
	inevitables. Es equivalente al nivel de bucles.
\end{itemize}

\subsubsection*{Dependencias de datos}

Se dice que un bloque de código B depende de un bloque de código A si ambos hacen referencia a las mismas
posiciones de memoria (variables) y B se encuentra después de A. Las dependencias de datos se clasifican
en tres categorías:
\begin{description}
	\item [RAW] Este tipo de dependencia de datos, cuyas siglas significan ``lectura tras escritura" 
	(\textbf Read \textbf After \textbf Write en inglés), ocurre cuando una operación necesita un dato que
	se calcula con otra operación. A esta categoría también se le llama \textit{dependencia verdadera}. Un
	ejemplo de esta dependencia sería el siguiente:
	\begin{gather*}
		a = b + c\\
		d = a + c
	\end{gather*}
	En este caso, $a$ se necesita en la segunda operación, pero se calcula en la primera, lo que
	hace imposible que ese trozo de código se ejecute en paralelo, puesto que podría ocurrir que se usase
	como operando sin tener el valor de primera operación.
	\item [WAW] Los datos que presentan esta dependencia, llamada ``escritura tras escritura" (\textbf
	Write \textbf After \textbf Write en inglés), son aquellos en los que dos operaciones guardan su
	resultado en el mismo dato. Esta dependencia es también denominada \textit{dependencia de salida}. Un
	ejemplo de esta dependencia es:
	\begin{gather*}
		a = b + c\\
		a = c + d
	\end{gather*}
	Se puede observar que en este ejemplo, la variable $a$ se usa para guardar el resultado de dos
	operaciones distintas. Por tanto, si se realizara paralelización, podría ocurrir que se ejecutara
	primero la segunda operación, cambiando el valor esperado de $a$.
	\item [WAR] Los datos con dependencias que entran dentro de esta categoría, denominada ``escritura tras
	lectura" (\textbf Write \textbf After \textbf Read en inglés), presentan el problema de que una
	operación usa un dato que después es utilizado para almacenar el resultado de otra operación. Esta
	dependencia es también denominada \textit{antidependencia}. Un ejemplo de esta dependencia podría ser
	el siguiente:
	\begin{gather*}
		b = a + c\\
		a = c + d
	\end{gather*}
	La variable $a$ es un operando de la primera operación y luego es utilizada para almacenar el
	resultado de la segunda operación. En caso de paralelización, pudiera ocurrir que el valor de $a$ en
	la primera operación fuera el resultado de la segunda operación.
\end{description}
De estas tres categorías, las dos últimas, WAW y WAR, son fácilmente resolubles renombrando variables.
Así, el ejemplo de la dependencia de salida podría ser resuelto de la siguiente manera:
\begin{gather*}
	a = b + c \quad\to\quad a_1 = b + c\\
	a = c + d \quad\to\quad a_2 = c + d
\end{gather*}
De esta forma, aunque aumentemos la memoria que necesita el programa, conseguimos que pueda ejecutarse de
forma paralela.

\end{document}