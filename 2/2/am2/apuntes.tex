\documentclass[11pt,twoside,titlepage,a4paper]{article}
%%%%%%%%%%%%%%%%%%%%%%%%%%%%%%%%%%%%%%%%%
%				PAQUETES				%
%%%%%%%%%%%%%%%%%%%%%%%%%%%%%%%%%%%%%%%%%
\usepackage{amsfonts}
\usepackage{amsmath}
\usepackage{amsthm}
\usepackage[spanish]{babel}
\usepackage{fancyhdr}
\usepackage[a4paper]{geometry}
\usepackage{mathtools}

%%%%%%%%%%%%%%%%%%%%%%%%%%%%%%%%%%%%%%%%%
%				MÁRGENES				%
%%%%%%%%%%%%%%%%%%%%%%%%%%%%%%%%%%%%%%%%%
\geometry{
	left=2.5cm,
	right=2.5cm,
	bottom=2.5cm
}

%%%%%%%%%%%%%%%%%%%%%%%%%%%%%%%%%%%%%%%%%
%				COMANDOS				%
%%%%%%%%%%%%%%%%%%%%%%%%%%%%%%%%%%%%%%%%%
\renewcommand{\contentsname}{Índice}
\theoremstyle{plain}
\newtheorem{teo}{Teorema}[section]
\newtheorem{prop}{Proposición}[section]
\theoremstyle{definition}
\newtheorem{defi}{Definición}[section]
\newtheorem{ej}{Ejemplo}[section]
\theoremstyle{remark}
\newtheorem*{obs}{Observación}

%%%%%%%%%%%%%%%%%%%%%%%%%%%%%%%%%%%%%%%%%
%		ENCABEZADO/PIE DE PAGINA		%
%%%%%%%%%%%%%%%%%%%%%%%%%%%%%%%%%%%%%%%%%
\setlength{\headheight}{14pt}
\pagestyle{fancy}
\fancyhf{}
\fancyhead[LE,LO]{Análisis Matemático II}
\fancyhead[RE,RO]{DGIIM UGR}
\fancyfoot[LE,RO]{\thepage}
\fancyfoot[RE,LO]{curso 2018-2019}

\begin{document}
%%%%%%%%%%%%%%%%%%%%%%%%%%%%%%%%%%%%%%%%%
%				 TÍTULO 				%
%%%%%%%%%%%%%%%%%%%%%%%%%%%%%%%%%%%%%%%%%
\title{\Huge{\textbf{Análisis Matemático II.\\Apuntes de clase}}}
\author{\textit{Manuel Gachs Ballegeer}}
\date{Doble Grado de Ingeniería Informática y Matemáticas\\2019}
\maketitle

%%%%%%%%%%%%%%%%%%%%%%%%%%%%%%%%%%%%%%%%%
%				 ÍNDICE 				%
%%%%%%%%%%%%%%%%%%%%%%%%%%%%%%%%%%%%%%%%%
\tableofcontents
\clearpage

%%%%%%%%%%%%%%%%%%%%%%%%%%%%%%%%%%%%%%%%%
%		 SUCESIONES DE FUNCIONES		%
%%%%%%%%%%%%%%%%%%%%%%%%%%%%%%%%%%%%%%%%%
\section{Sucesiones de funciones}

Sea $A\in\mathbb{R}^N$ un conjunto y $f_n:A\to\mathbb{R}^M,\;\forall n\in\mathbb{N}$ un conjunto
de funciones. Se define como sucesión de funciones a la sucesión $\{f_n\}$.

\subsection{Convergencia de sucesiones de funciones}

\begin{defi}
	Dada una sucesión $\{f_n\}$, se dice que \textit{converge puntualmente} a $f:A\to\mathbb{R}^M$,
	si $\{f_n(x)\}\to f(x)$ en el sentido usual de convergencia
\end{defi}
\begin{ej}
	$f_n:\mathbb{R}\to\mathbb{R},\; f_n(x)=\frac xn\quad\forall x\in\mathbb{R},\forall n\in\mathbb{N}$.
	\\Dado $x\in\mathbb{R}$, la sucesión $\{f_n\}=\{\frac xn\}\to 0$, esto es, $\{f_n\}$ converge
	puntualmente en $\mathbb{R}$, con límite $f\equiv 0$.
\end{ej}
\begin{ej}
	$f_n:\mathbb{R}\to\mathbb{R},\; f_n(x)=n^x\quad\forall x\in\mathbb{R},\forall n\in\mathbb{N}$.
	\\En particular, $\{f_n\}$ no converge puntualmente en $x\in]0,+\infty[$ \quad($\{f_n\}\to+\infty$)
	\\En particular, $\{f_n\}$ no converge puntualmente en $x\in]-\infty,0[$ \quad($\{f_n\}\to-\infty$)
	\\En particular, $\{f_n\}$ converge puntualmente en $x=0$ \quad($\{f_n\}\to 0$)
\end{ej}
\begin{ej}
	$f_n:[0,1]\to\mathbb{R},\; 
	f(x)= \left\{\begin{array}{lcc}
		0 & si & x \geq \frac 1n \\
		1-nx & si & 0\leq x < \frac 1n
	\end{array}\right.\quad\forall x\in[0,1],\forall n\in\mathbb{N}$
	\\Si $x\in]0,1]$, entonces $\exists n_0\in\mathbb{N}$ suficientemente grande tal que 
	$x\geq\frac{1}{n_0}$. Esto es, $f_{n_0}(x)=0$, y así $\{f_n(x)\}\to 0$.
	\\Si $x=0$, entonces $f_n(0)=1\;\forall n\in\mathbb{N}$, entonces $\{f_n(0)\}\to 1$.
	\\Por tanto, su límite sería una $f(x)$ tal que $f(x)=\left\{\begin{array}{lcc}
		1 & si & x=0 \\
		0 & si & x\in]0,1]
	\end{array}\right.$
\end{ej}
\begin{obs}
	La convergencia puntual no conserva la continuidad, y esto se ilustra en el ejemplo anterior,
	en el que el límite no es continuo pero las funciones sí lo son.
\end{obs}

\subsection{Convergencia uniforme}

\begin{defi}
	Sean $A\in\mathbb{R}^N$, $f_n:A\to\mathbb{R}^M\;\forall n\in\mathbb{N}$ y sea $f:A\to\mathbb{R}^M$.
	Se dice que $\{f_n\}$ converge uniformemente a $f$ si $\forall\varepsilon>0,\:\exists n_\varepsilon
	\in\mathbb{N}$ tal que
	\begin{equation*}
	||f_n(x)-f(x)||<\varepsilon\quad\forall n\in\mathbb{N},n\geq n_\varepsilon,\forall x\in A
	\end{equation*}
\end{defi}
%%%%%%%%%%%%%%%%%%%%%%%%%%%%%%%%%%%%%%%%%
%			FIN DEL DOCUMENTO			%
%%%%%%%%%%%%%%%%%%%%%%%%%%%%%%%%%%%%%%%%%
\end{document}