% Documento hecho a partir de la plantilla de LaTeX: https://github.com/Manuelbelgicano/DGIIM/blob/master/extra/plantilla_apuntes_v2.tex

\documentclass[11pt,twoside,titlepage,a4paper]{article}

%%%%%%%%%%%%%%%%%%%%%%%%%%%%%%%%%%%%%%%%%
%			   COLORINES				%
%%%%%%%%%%%%%%%%%%%%%%%%%%%%%%%%%%%%%%%%%
\usepackage{xcolor}
\definecolor{rojooscuro}{HTML}{8A0808}
\definecolor{burdeos}{HTML}{610B0B}
\definecolor{rojomorado}{HTML}{B40431}

%%%%%%%%%%%%%%%%%%%%%%%%%%%%%%%%%%%%%%%%%
%			  MATEMÁTICAS				%
%%%%%%%%%%%%%%%%%%%%%%%%%%%%%%%%%%%%%%%%%
\usepackage{amsmath} % Matemáticas
\usepackage{amsfonts} % Letras caligráficas para matemáticas
\usepackage{mathtools} % Matemáticas extra
\usepackage{amssymb} % Símbolos extra
\usepackage{amsthm} % Teoremas
% Eliminar '*' para añadir numeración a los lemas, teoremas...
\theoremstyle{definition}
\newtheorem*{defi}{Definición} % Comando para las definiciones
\newtheoremstyle{plain_rojo}{}{}{}{}{\color{rojooscuro}\bfseries}{:}{ }{}
\theoremstyle{plain_rojo}
\newtheorem*{lem}{Lema} % Comando para los lemas
\newtheorem*{teo}{Teorema} % Comando para los teoremas
\theoremstyle{remark}
\newtheorem*{cor}{Corolario} % Comando para los corolarios
\renewenvironment{proof}{{\bfseries\color{rojooscuro}Demostración:}}{\qed} % Cambiar el título de las demostraciones

%%%%%%%%%%%%%%%%%%%%%%%%%%%%%%%%%%%%%%%%%
%			   TIPOGRAFÍA				%
%%%%%%%%%%%%%%%%%%%%%%%%%%%%%%%%%%%%%%%%%
\usepackage{CrimsonPro}
\usepackage[T1]{fontenc}
% The font package uses mweights.sty which has som issues with the
% \normalfont command. The following two lines fixes this issue.
\let\oldnormalfont\normalfont
\def\normalfont{\oldnormalfont\mdseries}

%%%%%%%%%%%%%%%%%%%%%%%%%%%%%%%%%%%%%%%%%
%				  CÓDIGO				%
%%%%%%%%%%%%%%%%%%%%%%%%%%%%%%%%%%%%%%%%%
\usepackage{listingsutf8}
\lstset{
	inputencoding=utf8/latin1, % Codificación
	xleftmargin=1em, % Margen extra a la izquierda
	breaklines=true, % Romper líneas largas
	language=C++, % Lenguaje del código
	frame=single, % Enmarcado
	numbers=left, % Números de línea
	numbersep=8pt, % Separación de los números de línea
	tabsize=4, % Tamaño de los tabs
	frame=leftline, % Posición del enmarcado
	framerule=2pt, % Grosor del enmarcado
	showstringspaces=false, % Mostrar los espacios en las cadenas de caracteres
	basicstyle=\footnotesize\ttfamily, % Estilo del código
	keywordstyle=\color{burdeos}, % Estilo de las palabras reservadas
	numberstyle=\normalfont, % Estilo de los números de línea
	rulecolor=\color{rojooscuro}, % Estilo del enmarcado
	commentstyle=\color{red}, % Estilo de los comentarios
	stringstyle=\color{rojomorado} % Estilo de las cadenas de caracteres
}

%%%%%%%%%%%%%%%%%%%%%%%%%%%%%%%%%%%%%%%%%
%				MÁRGENES				%
%%%%%%%%%%%%%%%%%%%%%%%%%%%%%%%%%%%%%%%%%
\usepackage[a4paper]{geometry}
\geometry{
	left=2.5cm, % Margen izquierdo
	right=2.5cm, % Margen derecho
	bottom=2.5cm % Margen inferior
}

%%%%%%%%%%%%%%%%%%%%%%%%%%%%%%%%%%%%%%%%%
%  			  LISTAS/TABLAS				%
%%%%%%%%%%%%%%%%%%%%%%%%%%%%%%%%%%%%%%%%%
\usepackage{enumitem} % Opciones de personalización de listas
\renewcommand{\arraystretch}{1.3} %Cambiar el tamaño entre líneas de una tabla

%%%%%%%%%%%%%%%%%%%%%%%%%%%%%%%%%%%%%%%%%
%		COMANDOS PERSONALIZADOS 		%
%%%%%%%%%%%%%%%%%%%%%%%%%%%%%%%%%%%%%%%%%
\newcommand{\autores}{ % Autores del documento
	\begin{tabular}{l}
	Manuel Gachs Ballegeer
	\end{tabular}
}
\newcommand{\institucion}{ % Insitución
	Universidad de Granada
}
\newcommand{\infoextra}{ % Año o cualquier otra información para el título
	curso 2019-2020
}

%%%%%%%%%%%%%%%%%%%%%%%%%%%%%%%%%%%%%%%%%
%		ENCABEZADO/PIE DE PAGINA		%
%%%%%%%%%%%%%%%%%%%%%%%%%%%%%%%%%%%%%%%%%
\usepackage{fancyhdr}
\setlength{\headheight}{14pt}
\pagestyle{fancy}
\fancyhf{}
% Para que aparezca el título de la sección y no el número 
\renewcommand{\sectionmark}[1]{%
\markboth{#1}{}}
% Encabezado
\fancyhead[LE,RO]{\color{burdeos}{\leftmark}} % A la izquierda en pares, derecha en impares
\fancyhead[RE,LO]{\color{burdeos}{\institucion}} % A la derecha en pares, izquierda en impares
% Pie de página
\fancyfoot[LE,RO]{\Large\textbf{\thepage}} % A la izquierda en pares, derecha en impares
\renewcommand{\headrulewidth}{0.5pt} % Grosor de la línea

%%%%%%%%%%%%%%%%%%%%%%%%%%%%%%%%%%%%%%%%%
%			   	TÍTULOS					%
%%%%%%%%%%%%%%%%%%%%%%%%%%%%%%%%%%%%%%%%%
\usepackage{titlesec}
\titleformat{\section} % Estilo de las secciones
{\color{rojooscuro}\Huge\bfseries}
{\color{rojooscuro}\thesection}{1em}{}

%%%%%%%%%%%%%%%%%%%%%%%%%%%%%%%%%%%%%%%%%
%		   	  MISCELÁNEO				%
%%%%%%%%%%%%%%%%%%%%%%%%%%%%%%%%%%%%%%%%%
\usepackage{pagecolor} % Colorear las portadas
\renewcommand{\contentsname}{Índice} % Cambiar el título del índice
\setlength\parindent{0pt} % Tamaño de la sangría
\usepackage{graphicx} % Imágenes
\usepackage{blindtext} % Texto de relleno (Se puede eliminar)

\begin{document}

%%%%%%%%%%%%%%%%%%%%%%%%%%%%%%%%%%%%%%%%%
%				 PORTADA 				%
%%%%%%%%%%%%%%%%%%%%%%%%%%%%%%%%%%%%%%%%%
\begin{titlepage}
	\newpagecolor{rojooscuro} % Color de la portada
	\parbox[t]{\textwidth}{
		\raggedright
		\color{white}{\LARGE{\textbf{\institucion}}} \\
		\textit{\infoextra}
	}
	\vfill
	\parbox[c]{\textwidth}{
		\color{white}{
			\fontsize{70pt}{70pt}{\textbf{Probabilidad}} \\
			\bigskip \\
			\fontsize{40pt}{40pt}{\emph{Formulario}}
		}
	}
	\vfill
	\parbox[t]{\textwidth}{
		\raggedright
		\color{white}{\Large{\autores}} \\
	}
\end{titlepage}
\restorepagecolor

%%%%%%%%%%%%%%%%%%%%%%%%%%%%%%%%%%%%%%%%%
%				 ÍNDICE 				%
%%%%%%%%%%%%%%%%%%%%%%%%%%%%%%%%%%%%%%%%%
\tableofcontents
\clearpage

%%%%%%%%%%%%%%%%%%%%%%%%%%%%%%%%%%%%%%%%%
%				 DOCUMENTO 				%
%%%%%%%%%%%%%%%%%%%%%%%%%%%%%%%%%%%%%%%%%

%%%%%%%%%%%%%%%%%%%%%%%%%%%%%%%%%%%%%%%%%
%  	   ESPACIOS DE PROBABILIDAD			%
%%%%%%%%%%%%%%%%%%%%%%%%%%%%%%%%%%%%%%%%%

\section{Espacios de probabilidad}

Propiedades de la probabilidad:
\begin{itemize}[noitemsep]
	\item $P(\emptyset)=0,\;P(\Omega)=1$
	\item $P(\bar{A})=1-P(A)\;\;\forall A\in\mathcal{A}$
	\item $P(\bigcup_{i=1}^{n}A_i)=\sum_{i=1}^{n}P(A_i)$ donde $A_i\cap A_j=\emptyset,\;\forall j\neq i$
	\item $P(\bigcup_{i=1}^{n}A_i)\leq\sum_{i=1}^{n}P(A_i)$
	\item Si $A\subseteq B$, entonces $P(A)\leq P(B)$
	\item $P(A\cup B)=P(A)+P(B)-P(A\cap B)$
	\item $P(A\cup B\cup C)=P(A)+P(B)+P(C)-P(A\cap B)-P(A\cap C)-P(B\cap C)+P(A\cap B\cap C)$
	\item $P(A-B)=P(A\cap\bar{B})=P(A)-P(A\cap B)$
	\item Leyes de De Morgan:
	\begin{itemize}
		\item $\overline{A\cup B}=\bar{A}\cap\bar{B}$
		\item $\overline{A\cap B}=\bar{A}\cup\bar{B}$
	\end{itemize}
	\item $P(\bigcap_{i=1}^{n}A_i)\geq 1-\sum_{i=1}^{n}P(\bar{A}_i)$ (Desigualdad de Boole)
\end{itemize}

\subsubsection*{Probabilidad condicionada}

Cálculo de la probabilidad condicionada:
$$P(B/A)=\displaystyle\frac{P(A\cap B)}{P(A)}$$
En caso de que los sucesos $A$ y $B$ sean independientes, $P(A/B)=P(A)$.

Cálculo de la probabilidad condicionada a la intersección de sucesos:
$$P(C/A\cap B)=\displaystyle\frac{P(A\cap B\cap C)}{P(A)P(B/A)}$$
Cálculo de la intersección de sucesos a partir de las probabilidades condicionadas:
$$P(A_1\cap A_2\cap\ldots\cap A_n)=P(A_1)P(A_2/A_1)P(A_3/A_2\cap A_1)\cdots P(A_n/A_{n-1}\cap\ldots\cap A_1)$$
Si los sucesos son independientes, entonces $P(A_1\cap A_2\cap\ldots\cap A_n)=P(A_1)P(A_2)\cdots P(A_n)$

\subsubsection*{Teorema de la probabilidad total}

$$P(B)=\displaystyle\sum_{i=1}^{n}P(A_1)P(B/A_i)$$

\subsubsection*{Teorema de Bayes}

$$P(A_i/B)=\displaystyle\frac{P(A_i)P(B/A_i)}{\sum_{i=1}^{n}P(A_i)P(B/A_i)}$$

\newpage
\section{Distribuciones discretas}

\subsection{Distribución degenerada $X\rightsquigarrow D(c)$}

Función masa de probabilidad:




\end{document}