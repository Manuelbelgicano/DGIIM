\documentclass[11pt]{article}

\usepackage{enumitem}
\usepackage[utf8]{inputenc}
\usepackage[spanish, es-tabla, es-lcroman, es-noquoting]{babel}	
\usepackage{amsmath}
\usepackage[a4paper]{geometry}
\geometry{
	left=2.5cm, % Margen izquierdo
	right=2.5cm, % Margen derecho
	bottom=2.5cm % Margen inferior
}


\setlength{\parindent}{0in}
\everymath{\displaystyle} % Para que las fracciones y esas cosas se vean siempre grandes

%Gummi|065|=)
\title{\textbf{Problema 9}\\\Large{Relación Tema 4}}
\author{Miguel Ángel Fernández Gutiérrez\\
		Manuel Gachs Ballegeer\\
		Daniel Gonzálvez Alfert}
\date{}
\begin{document}

\maketitle

\section*{Enunciado}

Para la densidad de probabilidad condicionada siguiente:
$$
f_{Y/X=x}(y)=\left\{
	\begin{array}{rl}
		\frac{1}{x^2},  & \mbox{si } y \in [0,x^2] \\
		0, & \mbox{en otro caso }
	\end{array}
\right.
$$

Considerando que $X$ es una variable aleatoria continua con función de densidad de probabilidad:
$$
f_X(x)=\left\{
	\begin{array}{rl}
		3x^2,  & \mbox{si } x \in [0,1] \\
		0, & \mbox{en otro caso }
	\end{array}
\right.
$$

Calcular:
\begin{enumerate}[label=(\alph*)]
	\item La función de densidad de probabilidad conjunta de $X$ e $Y$.
	\item La función de densidad de probabilidad marginal de $Y$.
	\item Las curvas de regresión de $Y/X$ y $X/Y$, así como los errores cuadráticos medios asociados.
	\item Las razones de correlación de $Y/X$ y $X/Y$.
	\item Las rectas de regresión de $Y/X$ y $X/Y$.
	\item El coeficiente de correlación lineal.
\end{enumerate}

\newpage
\section*{Resolución}

\textbf{(a) La función de densidad de probabilidad conjunta de $X$ e $Y$.}\\

La calculamos a continuación:
$$f_{(X,Y)}(x,y)=f_{Y/X=x}\cdot f_X(x)=\frac{1}{x^2}\cdot 3x^2=3\;\;\forall 0 \leq y \leq x^2 \leq 1$$

\vspace{0.5cm}
\textbf{(b) La función de densidad de probabilidad marginal de $Y$.}\\

La calculamos a partir de la función de densidad calculada anteriormente:
$$f_Y(y)=\int_{\sqrt{y}}^13dx=3(1-\sqrt{y})\;\;\forall y \in [0,1]$$

\vspace{0.5cm}
\textbf{(c) Las curvas de regresión de $Y/X$ y $X/Y$, así como los errores cuadráticos medios asociados.}\\

\framebox[1.1\width]{$Y/X$}

$$\text{E}[Y/X]=\int_0^{x^2}yf_{Y/X}dy=\int_0^{x^2}\frac{y}{x^2}dy=\left.\frac{y^2}{2x^2}\right]_{y=0}^{y=x^2}=\frac{x^2}{2}$$
$$\text{E}[\text{E}[Y/X]]=\int_0^1\text{E}[Y/X]3x^2dx=\int_0^1\frac{3x^4}{2}dx=\left.\frac{3x^5}{10}\right]_{x=0}^{x=1}=\frac{3}{10}$$
$$\text{E}[\text{E}[Y/X]^2]=\int_0^1\text{E}[Y/X]^23x^2dx=\int_0^1\frac{3x^6}{4}dx=\left.\frac{3x^7}{28}\right]_{x=0}^{x=1}=\frac{3}{28}$$
$$\text{Var}[\text{E}[Y/X]]=\text{E}[\text{E}[Y/X]^2]-\text{E}[\text{E}[Y/X]]^2=\frac{3}{175}$$
$$\text{E}[Y]=\int_0^1y\cdot 3(1-\sqrt{y})dy=\left(\frac{3y^2}{2}-\frac{6y^{\frac{5}{2}}}{5}\right]_{y=0}^{y=1}=\frac{3}{10}$$
$$\text{E}[Y^2]=\int_0^1y^2\cdot 3(1-\sqrt{y})dy=\left(y^2-\frac{6y^{\frac{7}{2}}}{7}\right]_{y=0}^{y=1}=\frac{1}{7}$$
$$\text{Var}[Y]=\text{E}[Y^2]-\text{E}[Y]^2=\frac{37}{700}$$
$$\text{ECM}=\text{E}[\text{Var}[Y/X]]=\text{Var}[Y]-\text{Var}[\text{E}[Y/X]]=\frac{1}{28}$$

\framebox[1.1\width]{$Y/X$}

$$f_{X/Y=y}(x)=\frac{f_{(X,Y)}(x,y)}{f_Y(y)}=\frac{3}{3(1-\sqrt{y}}=\frac{1}{1-\sqrt{y}}\;\;\forall y \in [0,1]$$
$$\text{E}[X/Y]=\int_{\sqrt{y}}^1x\cdot f_{X/Y}dx=\int_{\sqrt{y}}^1\frac{x}{1-\sqrt y}dx=\left.\frac{x^2}{2(1-\sqrt y)}\right]_{x=\sqrt y}^{x=1}=\frac{1+\sqrt y}{2}$$
$$\text{E}[X^2/Y]=\int_{\sqrt y}^1 x^2\cdot f_{X/Y} dx=\int_{\sqrt y}^1 \frac{x^2}{1-\sqrt y}dx=\left.\frac{x^3}{3(1-\sqrt y)}\right]_{x=\sqrt y}^{x=1}=\frac{(1-\sqrt y)^2}{3}$$
$$\text{Var}[X/Y]=\frac{(1-\sqrt y)^2}{3}-\frac{(1+\sqrt y)^2}{2} = -\frac{y+10\sqrt y +1}{6}$$
$$\text{E}[X]=\int_0^1x\cdot f_X dx=\int_0^13x^3dx=\left.\frac{3x^4}{4}\right]_{x=0}^{x=1}=\frac{3}{4}$$
$$\text{E}[X^2]=\int_0^1x^2\cdot f_X dx=\int_0^13x^4dx=\left.\frac{3x^5}{5}\right]_{x=0}^{x=1}=\frac{3}{5}$$
$$\text{Var}[X]=\text{E}[X^2]-\text{E}[X]^2=\frac{3}{80}$$
Continuar...

\vspace{0.5cm}
\textbf{(d) Las razones de correlación de $Y/X$ y $X/Y$.}\\

$$\eta^2_{Y/X}=\frac{\text{Var}[E[Y/X]]}{\text{Var}[Y]}=\frac{12}{37}$$
Hacer el otro.

\vspace{0.5cm}
\textbf{(e) Las rectas de regresión de $Y/X$ y $X/Y$.}\\

\begin{align*}
E[XY] &= \int_{0}^{1}\int_{0}^{x^2}xyf_{xy}(x,y)\,\mathrm{d}y\mathrm{d}x =
\int_{0}^{1}\int_{0}^{x^2}xy3\,\mathrm{d}y\mathrm{d}x =\\
&= \int_{0}^{1}\frac{xy^23}{2}\Bigg]_{0}^{x^2}\,\mathrm{d}x =
\int_{0}^{1}\frac{x^53}{2}\,\mathrm{d}x = \frac{x^63}{12}\Bigg]_{0}^{1} = \frac14
\end{align*}
\begin{equation*}
\mathrm{Cov}(X,Y)= E[XY]-E[X]E[Y] = \frac14 - \frac34\cdot\frac{3}{10} = \frac{1}{40}
\end{equation*}

\framebox[1.1\width]{$Y/X$}\\

\begin{equation*}
y(x) = E[Y] + \frac{\mathrm{Cov}(X,Y)}{\mathrm{Var}(X)}(X-E[X]) = \frac{3}{10} +
\frac{\frac{1}{40}}{\frac{3}{80}}(x-\frac34) = \frac{3}{10} + \frac23(x-\frac34) = 
\frac23x - \frac15
\end{equation*}
\textit{Recta de regresión: } $y(x) = \frac23x - \frac15$
\begin{equation*}
\mathrm{E.C.M.} = \mathrm{Var}(Y) - \frac{\mathrm{Cov}(X,Y)^2}{\mathrm{Var}(X)} =
\frac{37}{700} - \frac{(\frac{1}{40})^2}{\frac{3}{80}} = \frac{37}{700} - 
\frac{1}{60} = \frac{19}{525}
\end{equation*}
\textit{Error cuadrático medio: } $\frac{19}{525}$

\framebox[1.1\width]{$X/Y$}\\

\begin{equation*}
x(y) = E[X] + \frac{\mathrm{Cov}(X,Y)}{\mathrm{Var}(Y)}(Y-E[Y]) = \frac34 +
\frac{\frac{1}{40}}{\frac{37}{700}}(y-\frac{3}{10}) = \frac34 + \frac{35}{74}
(y-\frac{3}{10}) = \frac{35}{74}y + \frac{45}{74}
\end{equation*}
\textit{Recta de regresión: } $x(y) = \frac{35}{74}y + \frac{45}{74}$
\begin{equation*}
\mathrm{E.C.M.} = \mathrm{Var}(X) - \frac{\mathrm{Cov}(X,Y)^2}{\mathrm{Var}(Y)} =
\frac{3}{80} - \frac{(\frac{1}{40})^2}{\frac{37}{700}} = \frac{3}{80} - 
\frac{7}{592} = \frac{19}{740}
\end{equation*}
\textit{Error cuadrático medio: } $\frac{19}{740}$

\vspace{0.5cm}
\textbf{(f) El coeficiente de correlación lineal.}\\

\begin{equation*}
\rho_{X,Y} = \frac{\mathrm{Cov}(X,Y)}{\sqrt{\mathrm{Var}(X)\mathrm{Var}(Y)}} =
\frac{\frac{1}{40}}{\sqrt{\frac{3}{80}\cdot\frac{3}{700}}} = 0.5615
\end{equation*}
\end{document}
