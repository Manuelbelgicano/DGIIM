\documentclass[12pt,a4paper]{article}
%%%%%%%%%%%%%%%%%%%%%%%%%%%%%%%%%%%%%%%%%
%				PAQUETES				%
%%%%%%%%%%%%%%%%%%%%%%%%%%%%%%%%%%%%%%%%%
\usepackage{amsmath} % Matemáticas
\usepackage{amsfonts} % Letras caligráficas para matemáticas
\usepackage{fancyhdr} % Encabezados/pies de páginas
\usepackage[a4paper]{geometry} % Márgenes
\usepackage{mathtools} % Matemáticas extra
\usepackage{titlesec} % Títulos especiales

%%%%%%%%%%%%%%%%%%%%%%%%%%%%%%%%%%%%%%%%%
%				MATEMÁTICAS				%
%%%%%%%%%%%%%%%%%%%%%%%%%%%%%%%%%%%%%%%%%
\everymath{\displaystyle} % Para que las fracciones y esas cosas se vean siempre grandes

%%%%%%%%%%%%%%%%%%%%%%%%%%%%%%%%%%%%%%%%%
%				MÁRGENES				%
%%%%%%%%%%%%%%%%%%%%%%%%%%%%%%%%%%%%%%%%%
\geometry{
	left=2.5cm,
	right=2.5cm,
	bottom=2.5cm
}

%%%%%%%%%%%%%%%%%%%%%%%%%%%%%%%%%%%%%%%%%
%		ENCABEZADO/PIE DE PAGINA		%
%%%%%%%%%%%%%%%%%%%%%%%%%%%%%%%%%%%%%%%%%
\setlength{\headheight}{14pt}
\pagestyle{fancy}
\fancyhf{}
\fancyhead[RE,RO]{This is another head}
\fancyhead[LE,LO]{This is a head}
\fancyfoot[CE,CO]{\thepage}

%%%%%%%%%%%%%%%%%%%%%%%%%%%%%%%%%%%%%%%%%
%				COMANDOS				%
%%%%%%%%%%%%%%%%%%%%%%%%%%%%%%%%%%%%%%%%%
\setlength\parindent{0pt} % Tamaño de la sangría
\titleformat{\section} % Estilo de las secciones
{\large\bfseries}
{\thesection}{1em}{}
\begin{document}
%%%%%%%%%%%%%%%%%%%%%%%%%%%%%%%%%%%%%%%%%
%				 TÍTULO 				%
%%%%%%%%%%%%%%%%%%%%%%%%%%%%%%%%%%%%%%%%%
\title{\Huge{\textbf{Probabilidad}}}
\author{\textit{Contenidos teóricos del examen}}
\date{}
\maketitle

%%%%%%%%%%%%%%%%%%%%%%%%%%%%%%%%%%%%%%%%%
%				 DOCUMENTO 				%
%%%%%%%%%%%%%%%%%%%%%%%%%%%%%%%%%%%%%%%%%

\section{Cálculo de momentos de la distribución uniforme continua}
Momentos no centrados:
\[
m_k=E[X^k]=\int_a^b\frac{x^k}{b-a}dx=\frac{b^{k+1}-a^{k+1}}{(k+1)(b-a)}
\]
Momentos centrados:
\[
\mu_k=E[(X-m_1)^k]=\int_a^b\frac{(x-m_1)^k}{b-a}dx=\begin{dcases*}
0 & si $k\in\mathbb{N}$ es impar \\
\frac{(b-a)^k}{2^k(k+1)} & en cualquier otro caso
\end{dcases*}
\]
\section{Cálculo de las funciones generatrices de momentos de las distribuciones: Uniforme continua, Normal y Exponencial}
Uniforme continua:
\[
\]
Normal:
\[
\]
Exponencial:
\[
\]
\section{Enunciar propiedades de la función de distribución de un vector aleatorio}
\section{Demostración de la caracterización de independencia por conjuntos de Borel}
\section{Demostración de la reproductividad de la Binomial}
\section{Enunciar el teorema de multiplicación de esperanzas}
\section{Demostración del teorema de descomposición de la varianza}
\section{Enunciar propiedades de esperanza condicionada}
\end{document}