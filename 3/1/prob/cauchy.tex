\documentclass[12pt,a4paper]{article}
%%%%%%%%%%%%%%%%%%%%%%%%%%%%%%%%%%%%%%%%%
%				PAQUETES				%
%%%%%%%%%%%%%%%%%%%%%%%%%%%%%%%%%%%%%%%%%
\usepackage{amsmath} % Matemáticas
\usepackage{amsfonts} % Letras caligráficas para matemáticas
\usepackage{fancyhdr} % Encabezados/pies de páginas
\usepackage[a4paper]{geometry} % Márgenes
\usepackage{mathtools} % Matemáticas extra
\usepackage{titlesec} % Títulos especiales
\usepackage{amsthm} % Teoremas

%%%%%%%%%%%%%%%%%%%%%%%%%%%%%%%%%%%%%%%%%
%				MATEMÁTICAS				%
%%%%%%%%%%%%%%%%%%%%%%%%%%%%%%%%%%%%%%%%%
\renewenvironment{proof}{\textbf{Demostración:}}{\qed} % Cambiar el título de las demostraciones
\everymath{\displaystyle} % Para que las fracciones y esas cosas se vean siempre grandes

%%%%%%%%%%%%%%%%%%%%%%%%%%%%%%%%%%%%%%%%%
%				MÁRGENES				%
%%%%%%%%%%%%%%%%%%%%%%%%%%%%%%%%%%%%%%%%%
\geometry{
	left=2.5cm,
	right=2.5cm,
	bottom=2.5cm
}

%%%%%%%%%%%%%%%%%%%%%%%%%%%%%%%%%%%%%%%%%
%		ENCABEZADO/PIE DE PAGINA		%
%%%%%%%%%%%%%%%%%%%%%%%%%%%%%%%%%%%%%%%%%
\setlength{\headheight}{14pt}
\pagestyle{fancy}
\fancyhf{}
\fancyhead[RE,RO]{This is another head}
\fancyhead[LE,LO]{This is a head}
\fancyfoot[CE,CO]{\thepage}

%%%%%%%%%%%%%%%%%%%%%%%%%%%%%%%%%%%%%%%%%
%				COMANDOS				%
%%%%%%%%%%%%%%%%%%%%%%%%%%%%%%%%%%%%%%%%%
\setlength\parindent{0pt} % Tamaño de la sangría

\begin{document}
%%%%%%%%%%%%%%%%%%%%%%%%%%%%%%%%%%%%%%%%%
%				 TÍTULO 				%
%%%%%%%%%%%%%%%%%%%%%%%%%%%%%%%%%%%%%%%%%
\title{\Huge{\textbf{Desigualdad de Cauchy-Schwarz}}}
\author{\textit{Probabilidad}}
\date{}
\maketitle

%%%%%%%%%%%%%%%%%%%%%%%%%%%%%%%%%%%%%%%%%
%				 DOCUMENTO 				%
%%%%%%%%%%%%%%%%%%%%%%%%%%%%%%%%%%%%%%%%%

Sean $X$ e $Y$ variables aleatorias sobre $(\Omega,\mathcal{A},P)$, centradas,
con momento de orden dos finito. Se tiene la desigualdad de Cauchy-Schwarz:
\begin{equation*}
(E[XY])^2\leq E[X^2]E[Y^2]
\end{equation*}

\begin{proof}
Definimos primero la variable aleatoria $W$ como $W=(X-\alpha Y)^2$. Puede
verse que $W$ es no negativa para todo $\alpha\in\mathbb{R}$. Por tanto, 
tenemos: 
\begin{align*}
0\leq E[W] &= E[(X-\alpha Y)^2] = E[X^2+\alpha^2Y^2-2\alpha XY] =\\
&= E[X^2]+E[\alpha^2Y^2]-E[2\alpha XY] = E[X^2]+\alpha^2E[Y^2]-2\alpha E[XY]
\end{align*}
Si definimos ahora $f(\alpha)=E[W]$, sabemos que $f(\alpha)\geq 0$, para todo
$\alpha\in\mathbb{R}$. Tomando ahora $\alpha=\frac{E[XY]}{E[Y^2]}$, sustituimos
en $f(\alpha)$:
\begin{align*}
0\leq f\Big(\frac{E[XY]}{E[Y^2]}\Big) &= E[X^2]+\Big(\frac{E[XY]}{E[Y^2]}
\Big)^2E[Y^2]-2\frac{E[XY]}{E[Y^2]}E[XY] =\\
&= E[X^2]+\frac{E[XY]^2}{E[Y^2]}- 2\frac{E[XY]^2}{E[Y^2]} =\\
&= E[X^2]-\frac{E[XY]^2}{E[Y^2]} 
\end{align*}
\begin{equation*}
0\leq E[X^2]-\frac{E[XY]^2}{E[Y^2]}\Leftrightarrow (E[XY])^2\leq E[X^2]E[Y^2]
\end{equation*}
Además, se da la igualdad cuando $f(\alpha)=0$, esto es, cuando $X=\alpha Y$.

\end{proof}

\end{document}